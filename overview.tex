%===============================================================================
% Overview.
%===============================================================================

\maketitle

\begin{figure}[htb!]
  \begin{center}
  \href{http://creativecommons.org/licenses/by-nc-sa/3.0/}{%
    \includegraphics{img/by-nc-sa_eu.pdf}}
  \end{center}
  \end{figure}

\newpage
\tableofcontents
\newpage

\begin{itemize}
\item Officially, this is ``Programming in Unix II'' (NSWI138) but ``Advanced
Unix Programming'' is a more convenient name. It is supposed to pick up where
``Programming in Unix'' (NSWI015) left off, and show some areas outside of the
scope of the 1st lecture that (not just) a C programmer is usually going to hit
sooner or later anyway.
\item All information you will need should be on
\url{http://mff.devnull.cz/}, in\-clu\-ding the last version of this material.
\item Source code examples are on \url{http://mff.devnull.cz/pvu2/src/}
\item Assumptions:
  \begin{itemize}
  \item NSWI015 passed (``Programming in Unix''). Materials for that are
  on-line on \url{http://mff.devnull.cz/pvu/slides/} but they are only in
  Czech.
  \item Good knowledge of the C language.
  \item Operating Systems theory basics.
  \end{itemize}
\item This text and source code examples are under construction, see ChangeLog
on page \pageref{CHANGELOG} for more information.
\end{itemize}

{\large\bf Terms of Use}

\begin{itemize}
\item Source code: standard 3-clause BSD license (see
\url{http://blogs.oracle.com/chandan/entry/copyrights\_licenses\_and\_cddl\_illustrated}
for illustration of what it allows)
\item This material: Creative Commons: Attribution + Non-commercial + Share Alike
\end{itemize}

%===============================================================================
% Overview.
%===============================================================================

\section{Overview}

%===============================================================================
\subsection{What is this lecture about?}

\begin{itemize}
  \item this lecture should extend the knowledge gained in Programming in UNIX
  (SWI015)
  \item covers more advanced areas a common UNIX C programmer usually hits
  sooner or later
\end{itemize}

Obviously, it cannot cover everything, however what is discussed is tightly
connected with source code examples and hands-on experience.

%===============================================================================
\subsection{The lecture will cover...}

not neccessarily in the following order...

\begin{itemize}
  \item a few notes on testing your code
  \item how to efficiently debug user level programs
  \item working with terminals and pseudo terminals, writing terminal
  applications
  \item advanced network programming
  \item advanced thread programming, using a non-POSIX thread API
  \item advanced IPC and I/O
  \item secure programming or how to write your programs in a more secure way
  and how to avoid common pitfalls
\end{itemize}

%===============================================================================
\subsection{A few notes on source code files}

\begin{itemize}
\item most of them should work on Solaris, Linux, BSD/OS X
\item with POSIX standard in mind but sometimes we show system specific
features, e.g. Solaris Threads API
\end{itemize}
