%===============================================================================
% Testing slide(s).
%===============================================================================

\section{Testing}

\subsection{Why ?}
\begin{itemize}
  \item To deliver good quality product (otherwise you will lose
  customers, money or lifes). Read the article "History's Worst
  Software Bugs" on
  \url{http://www.wired.com/software/coolapps/news/2005/11/69355?currentPage=all}
  in Wired (by Simson Garfinkel) about the nature and dreadful consequences of
  some of the bugs in software.
  \item To gain good level of confidence that the code works as expected
  (so you can sleep without unrest)
  \item Not to harm your reputation (by delivering crap product)
  \item Not to break the existing functionality when adding new feature
    or substantially changing current implementation:
  \url{http://en.wikipedia.org/wiki/Regression\_testing}
  \item To test stuff others have produced, e.g. when doing assesment
  of the product, when taking over the code someone else has written, etc.
  \item etc.
\end{itemize}

\subsection{When ?}
\begin{itemize}
  \item During development (to make sure you're actually progressing,
  not going sideways or regressing) \texttt{(1)}
  \item During code review (to be sure your changes done to satisfy code
  reviewer's comments are sound)
  \item Before integration (after merging with mainstream gate)
  \item After integration (to be sure someone else's change
  will not break your integration)
\end{itemize}

\begin{itemize}
\item[(1)] To write pscp (parallel scp(1) script) pscp-test was used:
\url{http://blogs.oracle.com/janp/entry/speeding\_up\_ssh\_data\_transfer}
  \begin{itemize}
    \item otherwise it would not be possible to complete the script in such
    a short time and catch most of the corner cases.
  \end{itemize}
\end{itemize}

\Task{Structure shape}{testing/structure-shape/}{%
This source code is based on a real world code from a daemon
implementing IKE protocol (see RFC 2409). During protocol exchange, the 2 sides
need to agree on common set of properties. The key established during the
exchange is then used to protect internet traffic. Given the properties of
symmetric ciphers (see birthday attack) and limitation of exposure in case of
key compromise, the rekeying is performed every once in a while. This rekeying
should not disrupt the traffic flow, so it happens in steps. There are two
rekeying times - {\em soft} and {\em hard}. When the soft timer expires, new
key negotiation is kicked off. When the hard timer expires, the old key is
deleted. Thus, there should be ample time between soft and hard timer values for
the rekeying to succeed. There are two sets of timers - one for time and one for
the amount of transferred data. Each timer has its default value. The program
tries to produce a set of 4 values which comply to the set of rules.}{%
\begin{enumerate}
\item Compile: \texttt{make clean \&\& make}
\item See the code and try to run with different arguments: \\
\texttt{./shapeup 50 60 0 0} \\
\texttt{./shapeup 0 0 0 0} , etc.
\item Find how many bugs are there in the code (there are at least 5 however
I am pretty sure you will find even more).
\item Modify the program slightly to check for the 4 conditions using
\texttt{assert()} and write a shell script which will perform fuzz testing,
i.e. run the program with random (integer) values. The crashes induced by
the \texttt{assert()}s will make the problems apparent.
\item Now you know what types of errors are there in the program so you can
design test cases to prevent regressions.
\item Create public project on \url{http://github.com} and populate it with
Makefile, \texttt{struct.c} files.
\item Modify \texttt{struct.c} code so that when specifying \texttt{-t} option
on the command line, it will check whether the 4 conditions are met. If not,
the program will return 1, otherwise it will return 0. The modification of the
program should be done using refactoring so that the previous functionality is
preserved, i.e. running \texttt{./shapeup <num> <num> <num> <num>} will still
work and will always return 0.
\item Create couple of unit tests, each unit test will run the program with
the \texttt{-t} option with multiple different sets of values.
Add \texttt{test} target to the Makefile which will run the tests. Even if one
of the tests fails, the rest of tests should be still executed with overall
result as fail.
\item Integrate Travis CI with your GitHub repo. With each push to the
repository the \texttt{test} target will be invoked. Carefully note the
limitations of building in the Travis CI environment (operating system choices,
compiler modes and versions, etc.)
\item Start fixing the program so that it works. Proceed with small steps,
commit each change and observe Travis CI builds.
\item Once you are confident that all of the bugs are fixed, ask one of your
fellow students to review your code. Fix any bugs he may find.
\end{enumerate}
}

%===============================================================================
\subsection{Types of testing}

\begin{itemize}
  \item How ?
  \begin{itemize}
    \item unit test (for given problem)
    \begin{itemize}
      \item test just the specific change, e.g. to make sure the fix really
      addresses the problem.
    \end{itemize}
    \item stress test \texttt{(3)}
    \item automated test suite \texttt{(2)}
    \item Test bed - deploy the product in near-real-life scenario \texttt{(1)}
  \end{itemize}
\end{itemize}

\begin{itemize}
\item[(1)] The "eat our own food/fly our own planes" spirit - install
  the product and use it internally/by yourself/by your friends
  (better motivation for good quality). Gives additional
  code coverage (\url{http://en.wikipedia.org/wiki/Code\_coverage}).
\item[(2)] Automated test suite should be written during development and
  after that used on a regular basis - (at least before integration of a
  change and before release is shipped),
  can be run on multitude of hardware platforms, with different settings -
  this will give a matrix of what should/could be tested. Test suite can
  contain both functional and stress tests.
\item[(3)] Example: when writing multithreaded application/library it's
  useful to write a stress test which would try to break it by issuing many
  valid requests/calls in unpredictable fashion. Useful for detecting
  boundary conditions, race conditions etc. This also gives more testing
  coverage.
\end{itemize}

%===============================================================================
\endinput
