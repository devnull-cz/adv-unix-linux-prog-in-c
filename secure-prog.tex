%===============================================================================
% Secure Programming.
%===============================================================================

\section{Secure Programming}

\begin{itemize}
  \item Motivation/goal:
  \begin{itemize}
    \item learn from the mistakes of others
    \item know how to write secure code (approach, techniques)
    \item be aware of what could go wrong/sensitive areas
  \end{itemize}
  \item We are not dealing with:
  \begin{itemize}
    \item secure use of cryptography primitives
    \item side-channel attacks
    \item robust secure network protocol design
    \item exploit techniques and analysis \texttt{(1)}
  \end{itemize}
\end{itemize}


\begin{itemize}
  \item[(1)] while technically interesting topic (e.g. for more in depth
  understanding of call procedures, assembly, HW architecture and internals
  in general) this is certainly out of scope and not in line with the main
  goal of this chapter
\end{itemize}

%===============================================================================

\subsection{More secure programs}

\begin{itemize}
  \item Prevention: 
  \begin{itemize}
    \item use of secure functions (strlcpy/strlcat)
  \end{itemize}
  \item Making attacker's life harder
  \begin{itemize}
    \item random allocations (mmap, malloc, ld.so)
    \item protection of memory segments (non-executable stack) (1)
  \end{itemize}
  \item Impact minimization
  \begin{itemize}
    \item privilege revocation (ping)
    \item privilege separation (OpenSSH)
    \item isolation (chroot, systrace, virtualization)
    \item separate uids for each service (\texttt{\_ntp}, \texttt{\_snmpd}, ...)
    \item buffer overflow detection and reaction (canary in the stack)
  \end{itemize}
\end{itemize}


\begin{itemize}
\item[(1)] non-exec stack can be worked around by e.g. overwriting return
address pointer with address of \texttt{system()} routine in libc and
constructing a frame which contains \texttt{"/bin/sh"} as an argument.
\end{itemize}

%===============================================================================

\subsection{Generic rules}

\begin{itemize}
  \item Check input data thoroughly (and do the right thing if they do not fit)
  \item {\it Do not assume} (1) (but rather check to confirm)
  \item Focus on corner cases (and go through all of them in detail)
  \begin{itemize}
    \item this includes handling failures properly (2)
  \end{itemize}
\end{itemize}


\begin{itemize}
  \item[(1)] A quotation from a whiteboard in office of an engineer.
  \item[(2)] I.e. returning failure code all the way up, freeing unneeded
    resources and reacting on it
    \begin{itemize}
    \item e.g. memory leak in error path can lead to Denial Of Service (DoS) by
      injecting number of invalid requests. See CR XXX (n2cp leak).
    \end{itemize}
\end{itemize}

{\bf Task}:
\begin{enumerate}
  \item Intro: KSSL is in-kernel SSL proxy in OpenSolaris/Solaris. It had a
    interoperability bug related to SSL/TLS protocol implementation.
  \item problem statement: {\it "... we needlessly fail a client hello request
        that has other compression methods in addition to the mandatory null
        compression method. This should be allowed since the spec allows
        supporting only the mandatory CompressionMethod.null and ignoring
        any other methods in client hello."}
  \item see the webrev \texttt{XXX/client\_hello.webrev.1} with
    the proposed fix
    \begin{itemize}
    \item focus on \texttt{kssl\_handle\_client\_hello()}
    \end{itemize}
  \item input info:
     \begin{itemize}
     \item m-block structure (\texttt{mblk\_t *mp}) represents SSL Client
     Hello message
     \item \texttt{mp->b\_rptr} is the beginning (pointer to area where we
     can start reading), \texttt{b\_wptr} is the end (pointer to area
     where we can start writing)
     \item see the definition in \texttt{usr/src/uts/common/sys/stream.h}:
\begin{verbatim}
    363 /*
    364  * Message block descriptor
    365  */
    366 typedef struct	msgb {
    367 	struct	msgb	*b_next;
    368 	struct  msgb	*b_prev;
    369 	struct	msgb	*b_cont;
    370 	unsigned char	*b_rptr;
    371 	unsigned char	*b_wptr;
    372 	struct datab 	*b_datap;
    373 	unsigned char	b_band;
    374 	unsigned char	b_tag;
    375 	unsigned short	b_flag;
    376 	queue_t		*b_queue;	/* for sync queues */
    377 } mblk_t;
\end{verbatim}
  \end{itemize}
   \item the assignment:
   \begin{enumerate}
        \item find the problem with the proposed fix
        \begin{itemize}
          \item hint: it's handy to look into the TLSv1 spec RFC 2246
                  (sections 4.3, 7.4.1.2)
          \item use code review/inspection approach (reader role)
	  \item What would happen if someone exploited the problem ?
	    (see first point of this Task)
        \end{itemize}
	\item once the problem is clear, propose correct solution
          \begin{itemize}
	  \item compare your solution with \texttt{XXX/client\_hello.webrev.2}
          \end{itemize}
   \end{enumerate}
\end{enumerate}

%===============================================================================

\subsection{Checking where necessary}

\begin{itemize}
  \item checking return values is not only useful for graceful handling of
    errors but can also have security impact
  \item assuming a function does always the right thing can later undermine
    sensitive area of code
\end{itemize}


{\bf Example}: \texttt{rtld} local root exploit in FreeBSD 7.2
\begin{itemize}
  \item intro: in FreeBSD, when loading setuid/setgid programs into memory,
                  dynamic (run-time) loader scrubs the environment so it
		  does not contain insecure (read user supplied) variables.
		  (mainly \texttt{LD\_LIBRARY\_PATH} and \texttt{LD\_PRELOAD})
   \begin{itemize}
   \item this is done via \texttt{unsetenv()} defined in
     \url{http://www.freebsd.org/cgi/cvsweb.cgi/src/lib/libc/stdlib/getenv.c:unsetenv()}
   \end{itemize}
  \item see the code in:
        \url{http://www.freebsd.org/cgi/cvsweb.cgi/src/libexec/rtld-elf/rtld.c}
    \begin{itemize}
    \item function \texttt{\_rtld()} is the "main entry point for dynamic
       linking". It contains this section:
    \end{itemize}
\begin{verbatim}
    trust = !issetugid();

    ld_bind_now = getenv(LD_ "BIND_NOW");
    /* 
     * If the process is tainted, then we un-set the dangerous environment
     * variables.  The process will be marked as tainted until setuid(2)
     * is called.  If any child process calls setuid(2) we do not want any
     * future processes to honor the potentially un-safe variables.
     */
    if (!trust) {
        unsetenv(LD_ "PRELOAD");
        unsetenv(LD_ "LIBMAP");
        unsetenv(LD_ "LIBRARY_PATH");
        unsetenv(LD_ "LIBMAP_DISABLE");
        unsetenv(LD_ "DEBUG");
        unsetenv(LD_ "ELF_HINTS_PATH");
    }
\end{verbatim}
  \item let's look at GETENV(3) man page which describes the semantics of
    return values of \texttt{unsetenv()}:
\begin{verbatim}
     The unsetenv() function deletes all instances of the variable name
     pointed to by name from the list.

RETURN VALUES
...
     The setenv(), putenv(), and unsetenv() functions return the value 0 if
     successful; otherwise the value -1 is returned and the global variable
     errno is set to indicate the error.

     [EINVAL]           The function getenv(), setenv() or unsetenv() failed
                        because the name is a NULL pointer, points to an empty
                        string, or points to a string containing an ``=''
                        character.
...
     [EFAULT]           The functions setenv(), unsetenv() or putenv() failed
                        to make a valid copy of the environment due to the
                        environment being corrupt (i.e., a name without a
                        value).  A warning will be output to stderr with
                        information about the issue.
\end{verbatim}
    \item the first case is no-variable/no-value, the latter is
    variable/no-value.
\begin{verbatim}
/*
 * Unset variable with the same name by flagging it as inactive.  No variable is
 * ever freed.
 */
int
unsetenv(const char *name)
{
        int envNdx;
        size_t nameLen;

        /* Check for malformed name. */
        if (name == NULL || (nameLen = __strleneq(name)) == 0) {
                errno = EINVAL;
                return (-1);
        }

        /* Initialize environment. */
        if (__merge_environ() == -1 || (envVars == NULL && __build_env() == -1))
                return (-1);

        /* Deactivate specified variable. */
        envNdx = envVarsTotal - 1;
        if (__findenv(name, nameLen, &envNdx, true) != NULL) {
                envVars[envNdx].active = false;
                if (envVars[envNdx].putenv)
                        __remove_putenv(envNdx);
                __rebuild_environ(envActive - 1);
        }

        return (0);
}
\end{verbatim}
  \item it calls \texttt{\_\_merge\_environ()} which contains this code:
\begin{verbatim}
          if (origEnviron != NULL)
                for (env = origEnviron; *env != NULL; env++) {
                        if ((equals = strchr(*env, '=')) == NULL) {
                                __env_warnx(CorruptEnvValueMsg, *env,
                                    strlen(*env));
                                errno = EFAULT;
                                return (-1);
                        }
\end{verbatim}
	  \item the \texttt{*envp[]} array contains pointers to strings like:
	    \texttt{HOME=/home/user} , \texttt{TERM=xterm} and such
	    \begin{itemize}
	    \item so if we replace one of the env strings by value-less string
	      \texttt{unsetenv()} will bail out as above and will not unset the
	      variable
	    \end{itemize}
  \item compare the above with OpenSolaris approach of handling setuid/setgid
	    programs:
\url{http://src.opensolaris.org/source/xref/onnv/onnv-gate/usr/src/cmd/sgs/rtld/common/setup.c:setup()}
  \begin{itemize}
   \item \texttt{setup()} which contains the main rtld functionality calls
   \texttt{security()} which performs the setuid/setgid check:
\begin{verbatim}
     	/*
     	 * Determine whether we have a secure executable.
     	 */
     	security(uid, euid, gid, egid, auxflags);
\end{verbatim}
  \item \texttt{security()} defined in
\url{http://src.opensolaris.org/source/xref/onnv/onnv-gate/usr/src/cmd/sgs/rtld/common/util.c#security}
  only sets the per-process \texttt{RT\_FL\_SECURE} flag
  (lines 3464, 3474, 3481).
\begin{verbatim}
   3445 /*
   3446  * Determine whether we have a secure executable.  Uid and gid information
   3447  * can be passed to us via the aux vector, however if these values are -1
   3448  * then use the appropriate system call to obtain them.
   3449  *
   3450  *  -	If the user is the root they can do anything
   3451  *
   3452  *  -	If the real and effective uid's don't match, or the real and
   3453  *	effective gid's don't match then this is determined to be a `secure'
   3454  *	application.
   3455  *
   3456  * This function is called prior to any dependency processing (see _setup.c).
   3457  * Any secure setting will remain in effect for the life of the process.
   3458  */
   3459 void
   3460 security(uid_t uid, uid_t euid, gid_t gid, gid_t egid, int auxflags)
   3461 {
   3462         if (auxflags != -1) {
   3463                 if ((auxflags & AF_SUN_SETUGID) != 0)
   3464                         rtld_flags |= RT_FL_SECURE;
   3465                 return;
   3466         }
   3467 
   3468         if (uid == (uid_t)-1)
   3469                 uid = getuid();
   3470         if (uid) {
   3471                 if (euid == (uid_t)-1)
   3472                         euid = geteuid();
   3473                 if (uid != euid)
   3474                         rtld_flags |= RT_FL_SECURE;
   3475                 else {
   3476                         if (gid == (gid_t)-1)
   3477                                 gid = getgid();
   3478                         if (egid == (gid_t)-1)
   3479                                 egid = getegid();
   3480                         if (gid != egid)
   3481                                 rtld_flags |= RT_FL_SECURE;
   3482                 }
   3483         }
   3484 }
\end{verbatim}
  \item \texttt{LD\_PRELOAD} handling is done in \texttt{ld\_preload()}
  defined in
\url{http://src.opensolaris.org/source/xref/onnv/onnv-gate/usr/src/cmd/sgs/rtld/common/setup.c#ld\_preload}:
\begin{verbatim}
         /*
          * If this a secure application, then preload errors are
          * reduced to warnings, as the errors are non-fatal.
          */
         if (rtld_flags & RT_FL_SECURE)
                 rtld_flags2 |= RT_FL2_FTL2WARN;
         if (expand_paths(*clmp, ptr, &palp, AL_CNT_NEEDED,
             PD_FLG_EXTLOAD, 0) != 0)
\end{verbatim}
  \item \texttt{expand\_paths()} defined in
  \url{http://src.opensolaris.org/source/xref/onnv/onnv-gate/usr/src/cmd/sgs/rtld/common/paths.c#expand\_paths}
  contains the actual check:
\begin{verbatim}
          /*
           * If this a secure application, validation of the expanded
           * path name may be necessary.
           */
          if ((rtld_flags & RT_FL_SECURE) &&
              (is_path_secure(str, clmp, orig, tkns) == 0))
          	continue;
\end{verbatim}
      \begin{itemize}
	\item this is different approach (set flag and check where needed)
	  which allows bigger flexibility. (but the checks have to be in
	  the right places as opposed to scrub-before-doing-anything
	  approach used in FreeBSD)
	  \begin{itemize}
	  \item ld.so.1(1) SECURITY section has more details, section FILES
	    sums it up:
    \texttt{/lib/secure} and \texttt{/usr/lib/secure} are
         \texttt{LD\_PRELOAD} locations for secure applications.
    \texttt{/lib/secure/64} and \texttt{/usr/lib/secure/64} are
         \texttt{LD\_PRELOAD} locations for secure 64-bit applications.
	 \item the locations are empty by default and writable only by
	 \texttt{root}.
	 \end{itemize}
     \end{itemize}
    \end{itemize}
  \item References:
    \begin{itemize}
    \item \url{http://xorl.wordpress.com/2009/12/01/freebsd-ld\_preload-security-bypass/}
    \item \url{http://stealth.openwall.net/xSports/fbsd-rtld-full-package}
    \end{itemize}
\end{itemize}

%===============================================================================

\subsection{Function classification}

\begin{itemize}
\item Library functions can be divided into several equivalence classes
  according to how secure/usable they are
    \begin{itemize}
    \item Some functions are inherently insecure and cannot be safely used
      at all
    \item Other functions require great deal of attention to get the code
      right (correct+secure code)
    \item There are even some functions which offer consistent API/behavior
      and make it easier to work with corner cases
    \end{itemize}
\end{itemize}


\begin{itemize}
  \item categorized list of functions:
    \url{http://hub.opensolaris.org/bin/view/Community+Group+security/funclist}
  \item Secure programming guidance at OpenSolaris Security community
    \url{http://www.opensolaris.org/os/community/security/library/secprog}
  \item Secure programming presentation by Scott Rotondo
    \url{http://opensolaris.org/os/community/security/library/secprog/secure\_prog.pdf}
\end{itemize}

%===============================================================================

\subsection{Buffer overflows}

\begin{itemize}
  \item informal definition: writing past the end of a buffer
  \begin{itemize}
    \item comon cases:
    \begin{itemize}
      \item string buffers but can be any buffer space
      \item buffer on the stack but applies also for on the heap buffers
    \end{itemize}
  \end{itemize}
  \item the danger:
  \begin{itemize}
    \item the surroundings of the buffer being overflowed (stack)
    \begin{itemize}
      \item function return address (redirecting code flow) \texttt{(1)}
      \item frame pointer (altering code flow)
    \end{itemize}
  \item persistent system space (heap)
    \begin{itemize}
      \item FILE I/O space
    \end{itemize}
  \end{itemize}
\end{itemize}


\begin{itemize}
  \item What happens on overflow:

\begin{verbatim}
  high address -------------------
                 heap
               -------------------


               -------------------
                return address
               -------------------
                frame pointer
               -------------------

                local variables

               -------------------

                       |
                       | stack grows downwards
                       v


  low address --------------------
\end{verbatim}
  \item[(1)] Smashing The Stack For Fun And Profit, Phrack Volume Seven, Issue
  49: \url{http://www.phrack.com/issues.html?issue=49\&id=14\&mode=txt}
\end{itemize}

%===============================================================================

\subsection{Safe string manipulation 1/2}

\begin{itemize}
  \item no boundary check: gets(), strcpy(), strcat(), sprintf()/vsprintf() (1)
  \item watch out when using: strncpy()/strncat()
\end{itemize}


\begin{itemize}
\item[(1)] unsafe functions:
  \begin{itemize}
  \item gets() does not check boundaries at all
  \end{itemize}
\item potentionally unsafe / avoid:
  \begin{itemize}
  \item strcpy(), strcat()
    \begin{itemize}
    \item do not check boundaries at all
    \item strcpy() does not zero terminate on overflow (but strcat() does)
    \item possible to calculate the necessary dst size for strcat()
      but hard to follow the code
    \end{itemize}
  \item sprintf()/vsprintf()
    \begin{itemize}
    \item no boundary check
    \item possible approach: compute the buffer size + control the fmt string
    \end{itemize}
  \end{itemize}
\item potentionally unsafe / use with caution:
  \begin{itemize}
  \item scanf() family
    \item scanf() should not be called without limit
      \item \texttt{scanf("\%s", str);} is asking for buffer overflow
      \item \texttt{scanf("\%10s", str);} needs 11 characters buffer
         (+ terminating \texttt{'\\\\0'})
  \end{itemize}
  \item snprintf()/vsnprintf()
  \begin{itemize}
    \item return the number of characters necessary (without the zero), not the
      actual number of characters written to the buffer
      - watch out for constructions like this:
        p += snprintf(p, lenp, "...");
  \end{itemize}
  \item strncpy()
  \begin{itemize}
    \item does not zero terminate on overflow
    \item zero-fills the remainder of the dst buffer in no-overflow case (perf)
  \end{itemize}
  \item strncat()
  \begin{itemize}
    \item always terminates (even on overflow)
    \item is not intuitive - some arithmetics for the size argument is almost
      always needed
  \end{itemize}
\end{itemize}


%===============================================================================

\subsection{Safe string manipulation 2/2}

\begin{itemize}
  \item overflow aware+helping: string(3C): \\
    \texttt{size\_t \funnm{strlcpy}(char *\emph{dst}, const char
    *\emph{src}, size\_t \emph{dstsize});} \\
    \texttt{size\_t \funnm{strlcat}(char *\emph{dst}, const char
    *\emph{src}, size\_t \emph{dstsize});}
  \begin{itemize}
    \item always terminate with zero, even in case of buffer overflow
    \item do not zero-out the rest of the dst string
    \item easy truncation detection
    \item the \emph{dstsize} argument is the total space in \emph{dst}
    \item for \texttt{strlcat()} nothing happens if the dst string is longer
      than size
    \item \texttt{strlcat()} returns
      \texttt{min\{dstsize,strlen(dst)\}+strlen(src)}
  \end{itemize}
\end{itemize}


\begin{itemize}
  \item[(1)] unfortunately, \texttt{strlcpy()}/\texttt{strlcat()} are not
  available in Linux world
    \begin{itemize}
    \item the main reasons for rejection (for inclusion into GNU libc) seem
      to be:
      \begin{enumerate}
      \item the functions are non-standard
      \item risk of truncation is considered to be greater than risk of overflow
      \end{enumerate}
    \end{itemize}
    \item the exact reasons are not very clear because of the strong language
      used by RedHat developers who are opposing the idea:
    \begin{itemize}
      \item \url{http://sources.redhat.com/ml/libc-alpha/2000-08/msg00061.html}
      \item \url{http://sources.redhat.com/ml/libc-alpha/2000-08/msg00053.html}
        (links from \url{http://en.wikipedia.org/wiki/Strlcpy\#Criticism})
    \end{itemize}
    \item that's why the Task below is even more important when writing
      code which uses static arrays because to make the code work on Linux
      as well there are basically 2 choices:
      \begin{itemize}
        \item either use the unsecure/inconsistent functions (not recommended)
	\item attach strlcpy/strlcat implementation (just like many projects do)
      \end{itemize}
\end{itemize}

\begin{itemize}
\item ensuring zero termination with \texttt{strncpy()/strncat()}:
  \begin{itemize}
  \item pass \texttt{dstlen - 1} as length and terminate by hand:
    \begin{verbatim}
    dst[dstlen] = '\0';
    \end{verbatim}
    \begin{itemize}
    \item if the dst variable is static or \texttt{calloc()}'ed the zero
      termination is not necessary but the code is then harder to inspect
      for overflow problems
    \end{itemize}
  \end{itemize}
\item problems with using \texttt{strncat()}:
  \begin{itemize}
  \item the size passed is the free space available in the buffer (not the
    total size of the buffer)
  \item the size argument must not count the terminating zero
    (even though the function always terminates)
  \end{itemize}
\item check for buffer overflow/truncation:
\begin{verbatim}
       if (strlcpy(dst, src, dstsize) >= dstsize)
               return (-1);

       if (snprintf(buf, sizeof (buf), "%s", src) >= sizeof (buf))
		return (-1);
\end{verbatim}
  \begin{itemize}
  \item the 'equal' case is where the string fits but without the
     terminating zero
  \end{itemize}
  \item \texttt{strlcpy()} paper:
    \url{http://www.openbsd.org/papers/strlcpy-paper.ps}
\end{itemize}

{\bf Task}: Write a secure (detects and prevents buffer overflow) function 
          \funnm{foo()} which constructs a string using the
          folllowing rule. The size of
	  the destination string is passed to the function.
	  Use hard-coded value in the \texttt{main()} calling \funnm{foo()}
	  (use e.g. \texttt{MAXPATHLEN} def{}ine from \texttt{<sys/param.h>}).

          \begin{itemize}
	  \item prototype: \texttt{int \funnm{foo}(char *\emph{input},
	    char *\emph{dst}, size\_t \emph{dstlen});}
	  \item rule (+ means concatenation):
          \begin{verbatim}
            env("HOME") + "/" + argv[1] + "/.foorc"
          \end{verbatim}
            \begin{itemize}
	    \item which means \texttt{main()} will pass \texttt{argv[1]}
	    to \funnm{foo()} as \emph{input}.
            \end{itemize}
 	  \item write using:
          \begin{enumerate}
	    \item \texttt{strcpy()}/\texttt{strcat()} (optional, only for
	      the really patient)
            \item \texttt{strncpy()}/\texttt{strncat()}
            \item \texttt{strlcpy()}/\texttt{strlcat()}
          \end{enumerate}
	  \item The function should return the length of the string or
             in case of overflow a negative number which if placed in
	     \texttt{abs()} is equal to the number of missing characters
	     in the destination string.
	  \item Construct number of unit tests for all possible cases
	    (most importantly for corner cases) and verify that each of the
	    implementations works correctly.
          \end{itemize}


%===============================================================================

\subsection{Time of use versus time of check}

\begin{itemize}
  \item possible race conditions - a window between check and use
    where the actual object can be swapped/changed but the reference to
    the object remains the same.
  \item setuid/setgid program often needs to verify that once it switches
    to a user the user will be able to read a file
  \item access(2) does the check using real uid/gid
\end{itemize}


\begin{itemize}
  \item Example of insecure approach (assume \texttt{path} points to a
  file in directory writable by others):
\begin{verbatim}
    /* we're running with euid = 0, and ruid = userid != 0 */
    if (access(path, R_OK|W_OK) < 0)
            return -1;

    fd = open(path, O_RDWR);
    ...
\end{verbatim}
  \begin{itemize}
  \item access(2) resolves the path to a XXX and performs the check. After
    that the XXX object is free'd. open(2) does the lookup again XXX.
    Problem is that this time different object can be used - the operation
    of check and return is not atomic. Now the setuid application might be
    writing to different file.
  \item correct approach: switch to the user and try to open the file -> no
    race condition
\begin{verbatim}
  seteuid(getuid());
  if ((fd = open(path, O_RDWR)) == -1)
      err(1, "open")
  seteuid(0);
\end{verbatim}
  \end{itemize}
  \item similar problem with stat():
\begin{verbatim}
  /* make sure "path" is a regular file before opening */
  if (stat(path, &buf) == -1 || !S_ISREG(buf.st_mode))
     err(...);
  fd = open(path, O_RDONLY);
\end{verbatim}
  \begin{itemize}
  \item the solution:
\begin{verbatim}
  if ((fd = open(path, O_NONBLOCK|O_RDONLY)) == -1)
     err(...);
  if (fstat(fd, &buf) == -1 || !S_ISREG(buf.st_mode))
     err(...);
\end{verbatim}
  \end{itemize}
  \begin{itemize}
  \item \texttt{O\_NONBLOCK} will prevent blocking when opening a device
  \end{itemize}
\item other functions which take file descriptor as argument instead of
  filename: \texttt{fchmod()} (chmod(2)), \texttt{fchdir()} (chdir(2)),
            \texttt{fchroot()} (chroot(2)), ...
\end{itemize}


%===============================================================================

\subsection{Memory locking}

\begin{itemize}
  \item prevents a memory region from being swapped out to disk (1)
  \item mlock() is perfectly useful for password managers, for example
  \item it seems one must use mmap() on Solaris to ensure the correct
    memory alignment (2)
  \item plock() is more generic but less granular
    \begin{itemize}
      \item allows locking of text/data segments
    \end{itemize}
  \item mlockall() locks every current/future mapped page in the address space
    \begin{itemize}
      \item the same semantics as mlock()
      \item can do selective unlock with munlock()
    \end{itemize}
\end{itemize}


\begin{itemize}
\item[(1)] semantics:
  \begin{itemize}
  \item no fork() inheritance
    \begin{itemize}
    \item unless \texttt{MAP\_PRIVATE} mmap() flag is used
    \end{itemize}
  \item locking of the same page by multiple processes is reference counted
    \begin{itemize}
    \item the last process which unlocks results in the page being unlocked
    \end{itemize}
  \item single process locking of the same page is not nested
    \begin{itemize}
    \item i.e. single unlock of page locked multiple times does unlock
    \end{itemize}
  \end{itemize}
  \item[(2)] see \priklad{secure-prog/mlock.c}
\end{itemize}

%===============================================================================

\subsection{Temporary files handling}

\begin{itemize}
\item mktemp(3C): \texttt{char *\funnm{mktemp}(char *\emph{template})}
  \begin{itemize}
    \item should not be used because of race condition between check and create
  \end{itemize}
\item mkstemp(3C)/mkdtemp(3C):
  \texttt{int \funnm{mkstemp}(char *\emph{template})}
  \begin{itemize}
  \item secure, race condition-less variants - return file descriptor of the
      temporary file and replaces the template argument with filename
  \end{itemize}
\item XXX following symlinks
\end{itemize}


Example: mktemp(1) in shell script

\begin{verbatim}
TMPFILE=`mktemp /tmp/example.XXXXXX`
if [[ -z "$TMPFILE" ]]; then exit 1; fi
print "program output" >> $TMPFILE
\end{verbatim}

%===============================================================================

\subsection{Off-by-one errors / Integer overflows}

\begin{itemize}
\item \texttt{calloc()} can overflow internally (1)
\item XXX off-by-one
\end{itemize}


\begin{itemize}
  \item[(1)] when computing the size of the number of bytes to be
  allocated XXX
\end{itemize}


%===============================================================================

\subsection{Format string overflows}

\begin{itemize}
\item XXX user-supplied format string can be used to overflow stack of the
process
  - examples XXX:
    %n
    %x
\end{itemize}



%===============================================================================

\subsection{Privilege separation}

\begin{itemize}
  \item the goal: minimize the amount of code running as root/with privileges
  \item basic principle - one unprivileged process for input data processing,
		    another process for performing privileged tasks.
		    they communicate with each other using set of well
                    defined messages.
\end{itemize}


XXX picture


%===============================================================================

\subsection{Examples of privilege separated programs}

\begin{itemize}
\item OpenSSH:
  \begin{itemize}
  \item master process: accepts new connections, then immediately forks new
    process group (1)
  \item alternative privilege separation model (2)
  \end{itemize}
\item OpenBGPD
  \begin{itemize}
  \item parent: runs as root, enters routes into kernel
    - works with routing socket (3)
  \end{itemize}
\item OpenNTPD
  \begin{itemize}
  \item parent: runs as root, sets the time
  \item ntp engine: runs as non-root uid/gid \texttt{\_ntp:\_ntp} and
     chrooted to \texttt{/var/empty}
  \item simple communication (4)
  \end{itemize}
\end{itemize}


\begin{itemize}
  \item[(1)] process group consists of:
    \begin{itemize}
     \item monitor: privileged process, handles authentication
     \item listener: SSH communication over the network
    \end{itemize}
  \item[(2)] OpenSSH: XXX privsep protocol with XXX messages
    - SunSSH: altprivsep with XXX messages
  \item[(3)] children processes:
    \begin{itemize}
    \item child \#1 (session engine): manages sessions (TCP connections)
    - runs as unprivileged user \texttt{\_bgpd}, chroots to \texttt{/var/empty}
    \item child \#2 (route decision engine): works with routing tables
    - runs as unprivileged user \texttt{\_bgpd}, chroots to \texttt{/var/empty}
    \end{itemize}
  \item[(4)] \texttt{socketpair}, 2 message types (set the time, lookup
     hosts using the files in \texttt{/etc})
\end{itemize}

%===============================================================================

\subsection{Privilege revocation/bracketting}

\begin{itemize}
  \item use only the privileges which are needed (at the time they are needed)
  \item \texttt{seteuid()}
    \begin{itemize}
      \item better: \texttt{setreuid(getuid(), getuid());}
    \end{itemize}
  \item privilege awareness (privileges(5) on Solaris)
\end{itemize}



%===============================================================================

\subsection{Non-executable stack}

\begin{itemize}
\item XXX
\item XXX OpenBSD approach for platforms lacking N-X
\end{itemize}

\texttt{cc -M /usr/lib/ld/map.noexstk myprogram.c}



%===============================================================================

\subsection{Prevention techniques and tools}

\begin{itemize}
\item check return values of functions (and interpret + act upon them correctly)
  \begin{itemize}
    \item use \texttt{-Wall} and fix all warnings
  \end{itemize}
\item chroot(2) syscall
  \begin{itemize}
  \item close unneeded descriptors \bf{and} \texttt{chdir()} \bf{and}
   \texttt{chroot()}
  \end{itemize}
\item automated checking
  \begin{itemize}
  \item lint, Parfait, XXX
  \end{itemize}
\item stress testing (1)
\item peer code reviews
\item do the what-went-wrong case studies regularly (2)
\end{itemize}


\begin{itemize}
  \item[(1)] try to break the program (e.g. by feeding it all sorts of bogus
   data, injecting faults, etc.)
  \item[(2)] Analyses of vulnerabilities/fixes: \url{http://xorl.wordpress.com}
    (be sure to double check/do your research, though)
  \item References:
  \begin{itemize}
    \item CERT C secure coding standard:
      \url{https://www.securecoding.cert.org/}
    \item Secure Coding in C and C++ (Robert C. Seacord)
      \url{http://www.cert.org/books/secure-coding/}
  \end{itemize}
\end{itemize}

\endinput
