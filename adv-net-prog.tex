%===============================================================================
% Advanced Network Programming.
%===============================================================================

\section{Advanced Network programming}

beyond the "listen on a socket and accept requests"

%===============================================================================

\subsection{Accessing low level structures}

\begin{itemize}
  \item get a list of interfaces on the system (ala ifconfig(1M)) (1)
  \begin{itemize}
    \item OpenBSD: getifaddrs(3) - get list of IPs on the system
    \item Solaris: ioctl() interface and \texttt{SIOCGLIFNUM},
      \texttt{SIOCGLIFCONF} (2)
  \end{itemize}
\end{itemize}


\begin{itemize}
  \item[(1)] e.g. to listen on each individual address in the system
  \begin{itemize}
    \item handy when the reply has to have the right source address
      (e.g. NTP protocol)
    \item need to track down the changes while running (see the
      \texttt{PF\_ROUTE} socket slide)
  \end{itemize}
  \item[(2)] Generic approach of copying out sequence of structures from
    the kernel to userland:
\begin{enumerate}
  \item open \texttt{AF\_INET}/\texttt{SOCK\_DGRAM} socket
  \item fill the lifnum structure
  \item issue the \texttt{SIOCGLIFNUM} ioctl to get number of interfaces
  \item allocate buffer for number of entries acquired in previous step
  \item fill the lifconf structure
  \item issue the \texttt{SIOCGLIFCONF} ioctl to get interface structures
  \item traverse the list of lifreq structures
\end{enumerate}
  \item \texttt{lifreq} is defined in \texttt{usr/src/uts/common/net/if.h}
and contains various interface specific data:
\begin{itemize}
  \item name
  \item physical info
  \item address info
  \item MTU
\end{itemize}
  \item {\bf Example}: \priklad{adv-net-prog/ifclist/}
\begin{enumerate}
	\item compile
\begin{verbatim}
	   make clean && make
\end{verbatim}
	\item run for both address families
\begin{verbatim}
	   ./getlist 4
	   ./getlist 6
\end{verbatim}
	\item compare the output from the list of interfaces
\begin{verbatim}
	   ifconfig -a
\end{verbatim}
\end{enumerate}
{\bf Task}: adjust \texttt{ifclist} to print out IPv4/IPv6 addresses
  on the interfaces
\end{itemize}


%===============================================================================

\subsection{More setsockopt() options}

\begin{itemize}
  \item \texttt{SO\_SNDBUF}/\texttt{SO\_RCVBUF} - performance tuning (1)
  \item \texttt{SO\_EXCLBIND} - exclusive binding (2)
  \item \texttt{SO\_KEEPALIVE} (3)
  \item \texttt{SO\_DONTROUTE} - append ifc name and send to the wire (4)
  \item \texttt{IPPROTO\_IPV6} + \texttt{IPV6\_V6ONLY}
  \item \texttt{IPPROTO\_TCP} + \texttt{TCP\_NODELAY}
  \item \texttt{IP\_SEC\_OPT} - use IPsec per-socket (per-port) policy,
    including policy bypass (5)
\end{itemize}


%	- (janp) zajimavy priklad by byl ukazat na dummy-like netu, ze high
%	  latency, high bandwidth link benefituje hodne ze zvetseni TCP window
%	  atd.

\begin{itemize}
  \item some of the socket options are OS specific
  \item[(1)] we can set {snd,rcv}buf not only for TCP/UDP sockets but also
	  for ICMP (raw ICMP socket)
  \item[(2)] related to security (XXX find the SSH security bug with
    AF\_INET6/AF\_INET fallback)
  \item[(3)] mention OpenSSH and the distinction of Keepalive messages
	on application layer
%	XXX zminit, ze TCP keepalive je hlavne pro detekci crashu celych stroju.
  \item[(4)] gets full control over sending together with RAW sockets
	  - see ping(1M) and -r option
  \item[(5)]
  \begin{itemize}
    \item {\bf Example}: \priklad{adv-net-prog/setsockopt/policy-bypass.c}
\begin{enumerate}
		 \item compile
		 \item run the program w/out bypass
\begin{verbatim}
	    ./bypass 0 20.0.0.1 80
\end{verbatim}
		 \item set policy
\begin{verbatim}
	    echo "{ raddr 20.0.0.1 rport 80 } ipsec
	      { encr_auth_alg sha1 encr_alg aes }" > /tmp/mypolicy.conf
	    chmod 644 /tmp/mypolicy.conf
	    ipsecconf -f -a /tmp/mypolicy.conf
\end{verbatim}
	         \item run the program w/out bypass again
\begin{verbatim}
	    ./bypass 0 20.0.0.1 80
\end{verbatim}
		 \item run the program with bypass
\begin{verbatim}
	    ./bypass 1 20.0.0.1 80
\end{verbatim}
\end{enumerate}
  \item {\bf Example}: \priklad{adv-net-prog/setsockopt/auth.c}
		   \begin{itemize}
		   \item similar to policy-bypass.c but provides traffic 
                   authentication
		   \end{itemize}
		 \begin{enumerate}
		 \item setup IKE
		    XXX
                 \item initiate traffic
		 \end{enumerate}
  \item References:
    \url{http://blogs.oracle.com/danmcd/entry/put\_ipsec\_to\_work\_in}
  \end{itemize}  
\end{itemize}  
%XXX
%	- byl bych opravdu velmi opatrny s IPsecem, vetsina lidi nebude o tom
%	  skoro nic vedet. Napriklad ja o tom skoro nic nevim krome opravdu
%	  zakladnich veci.


%===============================================================================

\subsection{Raw sockets}

\begin{itemize}
  \item bypass packet encapsulation, i.e. push our own headers on the wire
    \begin{itemize}
      \item this is for layer-3/network control
      \item for layer-2/link access PF\_PACKET interface is needed (1)
    \end{itemize}
  \item \texttt{SOCK\_RAW} is a type of socket (see socket(3socket))
    \begin{itemize}
      \item can be used in various domains (will focus on PF\_INET)
    \end{itemize}
  \item need to know packet header format and protocol specifics (2)
\end{itemize}


%	XXX
%		- hmm, neni tomu moc rozumnet. Pouzivani "need to know" se mi
%		  zda trochu nejasny. Melo by tam spis byt "you must know..."
%		  nebo "programmer must know" nebo tak. "need to know" se plete
%		  s klasifikaci utajeni dokumentu.

\begin{itemize}
  \item[(1)] works on Linux, Solaris
  \item[(2)] complete+correct IP header will be needed for most of the cases
    \begin{itemize}
    \item this means we have to compute the checksum
    \item we're pushing stuff to the wire, hence correct byte ordering is
        necessary (\texttt{htons()} etc, see byteorder(3socket))
    \end{itemize}
  \item[(3)] if we need application layer protocols we can let the stack to
    build IP header for us
    \begin{itemize}
    \item ICMP
\begin{verbatim}
    socket(PF_INET, SOCK_RAW, IPPROTO_ICMP)
\end{verbatim}
    \item TCP
\begin{verbatim}
    socket(PF_INET, SOCK_RAW, IPPROTO_TCP)
\end{verbatim}
    \item UDP
\begin{verbatim}
    socket(PF_INET, SOCK_RAW, IPPROTO_UDP)
\end{verbatim}
    \end{itemize}
\end{itemize}

{\bf Example}: construct ICMP packet header with arbitrary type and code
	 \texttt{adv-net-prog/raw-socket}
\begin{enumerate}
	 \item compile
\begin{verbatim}
    make clean && make
\end{verbatim}
	 \item construct ICMP echo request
\begin{verbatim}
    ./icmpraw 20.20.10.10 8 0
\end{verbatim}
\end{enumerate}

%- OS specifics 
%  - endianess in header fields
%  - boundaries of what is allowed to push

\begin{itemize}
\item libnet is a good C wrapper for raw socket manipulation
  \begin{itemize}
    \item \url{http://sourceforge.net/projects/libnet-dev/}
    \item \url{http://www.securityfocus.com/infocus/1386}
  \end{itemize}
\end{itemize}

%XXX
%	- chybi tu par odkazu na na RFC pro IP apod.


%===============================================================================

\subsection{Layer-2 (link) access}

\begin{itemize}
  \item possible choices for link layer access:
    \begin{itemize}
      \item Datalink Provider Interface (DLPI) - Solaris
      \item \texttt{SOCK\_PACKET} - Linux
      \item Berkeley Packet Filter (BPF) - almost everywhere
    \end{itemize}
  \item libpcap
    \begin{itemize}
      \item high-level interface to packet capture
        \item wrapper for all the 3 choices above - OS independent interface
      \item pcap(3)
    \end{itemize}
\end{itemize}


%	XXX
%		- chybi mi tu kratka motivace. Napr. ze chci cist, co na siti
%		  bezi na port 80 atd., at si pod tim muzou neco predstavit.
%		- myslim ze by se hodil i obrazek, jak to obecne cely funguje.

\begin{itemize}
  \item Sequence of steps:
  \begin{enumerate}
    \item find capturing device: pcap\_lookupdev() (optional)
    \item get packet capture descriptor and open the device for capture:
      pcap\_open\_live()
      - alternatively open a file with stored traffic: pcap\_open\_offline()
    \item get network address and mask for the interface: pcap\_lookupnet()
      - will be needed for compiling the filter
    \item compile the filter: pcap\_compile()
    \item put the filter into effect: pcap\_setfilter()
    \item read the data: pcap\_loop()/pcap\_next()
  \end{enumerate}
  \item {\bf Example}: how to integrate pcap into an application
	 \priklad{adv-net-prog/packet-capture}
	 Capture and interpret TCP segements:
  \begin{enumerate}
	 \item compile
\begin{verbatim}
	    make clean && make
\end{verbatim}
	 \item run
\begin{verbatim}
	    ./tcptraffic host www.ms.mff.cuni.cz and port 80
\end{verbatim}
	 \item initiate some traffic
\begin{verbatim}
	    echo "GET /" | nc -w 10 www.ms.mff.cuni.cz 80
\end{verbatim}
	 \item see the output
  \end{enumerate}
  \item Resources:
    \begin{itemize}
      \item UNIX network programming, volume 1 (Richard W. Stevens),
         29 Data link access
    \end{itemize}
\end{itemize}



%===============================================================================

\subsection{Porting applications to IPv6}

\begin{itemize}
  \item or alternatively - writing portable applications
  \item or better - address family agnostic applications
  \item AF-specific data structures (avoid if possible (1)):
    \begin{itemize}
      \item \texttt{AF\_INET} (IPv4): \texttt{sockaddr\_in},
         \texttt{in\_addr\_t}
      \item \texttt{AF\_INET6} (IPv6): \texttt{sockaddr\_in6},
         \texttt{in6\_addr\_t}
    \end{itemize}
  \item AF-agnostic data structures:
    \begin{itemize}
      \item \texttt{sockaddr}, \texttt{sockaddr\_storage} (2)
    \end{itemize}
  \item AF-specific functions (avoid):
    \begin{itemize}
      \item \texttt{gethostbyname()}, \texttt{gethostbyaddr()}
    \end{itemize}
  \item AF-agnostic functions:
    \begin{itemize}
      \item \texttt{getaddrinfo()}, \texttt{getnameinfo()}
    \end{itemize}
\end{itemize}


% XXX obrazek datovych struktur s pretypovanim

% idea: kdyz zacinam psat novy sitovy program, tak co je potreba mit na
%       pameti a udelat, aby aplikace byla rovnou funkcni i pro IPv6.

\begin{itemize}
  \item[(1)] Sometimes it is not possible to avoid AF-specifics
  \item[(2)] de{}fine in \texttt{/usr/include/sys/socket\_impl.h} (will be
  included with \texttt{<sys/socket.h>})
\end{itemize}

{\bf Example}: Watch out when listening on \texttt{AF\_UNSPEC} sockets:
\begin{itemize}
  \item CVE-2008-1483 in SunSSH (CR 6684003)
  \item \url{http://src.opensolaris.org/source/history/onnv/onnv-gate/usr/src/cmd/ssh/libssh/common/channels.c}
  \item \texttt{x11\_create\_display\_inet()} does the setup for X11 forwarding
    over SSH channel
  \item it works like this: \priklad{adv-net-prog/ipv6-port-bind/cycles.c}
\begin{lstlisting}
    for port in (low, high); do
        list = resolve(port for all address faimilies)
        for family in list; do
            create socket for family
            bind the socket to the port and family
            if the bind fails
                if the list is non-empry
                    continue
                      (try the same port number with different address family)
                else
                    try next port number
                fi
            else
                record the port number
            fi
        done
    done
\end{lstlisting}
  \item resolving will produce list with IPv6 family entry first and IPv4
    entry second
  \item if someone binds the port to IPv6 \emph{only} then \texttt{bind()}
    for IPv4 will succeed
    \begin{itemize}
    \item the forwarder thinks everything is okay and passes the port number
    \end{itemize}
  \item the app will try to connect to that port
    \begin{itemize}
    \item some will try IPv6 first but that is hijacked by the attacker
      (who key steal keystrokes in case of X11 connection etc.)
    \end{itemize}
\end{itemize}

%   XXX add attacker app (use nc ?)


%===============================================================================

\subsection{PF\_ROUTE socket}

\begin{itemize}
\item exists on BSD, Linux, Solaris
  \begin{itemize}
    \item PF\_ROUTE exists since 4.3BSD-Reno
  \end{itemize}
  \item usage: routing daemons, network server programs which need to observe
        changes in local addresses and logical interfaces (adding/removing
        server instances)
\end{itemize}


\begin{itemize}
  \item Solaris man pages:
  \begin{itemize}
    \item routing(7P)
    \item route(7P) - describes \texttt{PF\_ROUTE} interface
  \end{itemize}
  \item working with the socket (passive mode):
\begin{enumerate}
  \item open new socket
\begin{verbatim}
     socket(PF_ROUTE, SOCK_RAW, 0);
\end{verbatim}
  \item enter event loop watching for new messages: select/poll + read
  \item processing messages
\end{enumerate}
\end{itemize}

{\bf Example}: write something like 'route monitor' (see route(1M)) but only
    for watching interface additions/removals
     - useful for managing per-address server instances (e.g. UDP servers)
%	 XXX - server instances? To nejak nechapu.
         \priklad{adv-net-prog/routing-socket/}
\begin{enumerate}
  \item compile
\begin{verbatim}
      make clean && make
\end{verbatim}
  \item run
\begin{verbatim}
      ./rtwatch
\end{verbatim}
  \item create new interface (has to plumb IPv6 interface first, then add logical)
\begin{verbatim}
      ifconfig iprb0 inet6 plumb up
      ifconfig iprb0 inet6 addif 2001:0db8:3c4d:55:a00:20ff:fe8e:f3ad/64 up
\end{verbatim}
  \item observe the messages emitted by rtwatch
\end{enumerate}


%===============================================================================

\subsection{PF\_KEY socket}

\begin{itemize}
  \item analogous to the PF\_ROUTE socket
  \item available in Solaris, BSD
    \begin{itemize}
      \item man page: pf\_key(7P)
    \end{itemize}
  \item manipulates IPsec Security Associations (SAs) (1) from userland
  \item app says ADD/DELETE/UPDATE SAs to the kernel
  \item \texttt{PF\_KEY} interface described in RFC 2367
  \item there is also \texttt{PF\_POLICY} socket on Solaris which manipulates
    security policy kernel entries
    \begin{itemize}
      \item which decide whether packet will be allowed, denied or changed (2)
    \end{itemize}
\end{itemize}


\begin{itemize}
  \item[(1)] SAs src/dst description, contain key material, timing information
    etc.
    \begin{itemize}
      \item SA is unidirectional
      \item they are used for computing authentication data and/or encrypt
        packets on the wire
    \end{itemize}
  \item[(2)] simple per-port (per-application) firewall
\end{itemize}

{\bf Example}: flush IPsec AH SAs
	 \priklad{adv-net-prog/key-socket/}
\begin{enumerate}
	 \item compile the program
\begin{verbatim}
	    make clean && make
\end{verbatim}
  	 \item Use Example \priklad{adv-net-prog/setsockopt/auth.c} to set IPsec SAs
         \item observe SAs in the system and start watching
\begin{verbatim}
	    ipseckey dump
	    ipseckey monitor
\end{verbatim}
	 \item flush the AH SAs using our program
\begin{verbatim}
	    ./ahflush
\end{verbatim}
         \item check
\begin{verbatim}
	    ipseckey dump
\end{verbatim}
\end{enumerate}

\begin{itemize}
  \item References:
    \url{http://blogs.oracle.com/danmcd/entry/pf\_key\_in\_solaris\_or}
\end{itemize}


\endinput
